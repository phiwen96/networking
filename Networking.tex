\documentclass{article}
\usepackage[utf8]{inputenc} % - Defines what coding LaTeX uses. Use this one.
\usepackage[swedish]{babel}
\usepackage{graphicx} % - Package for including images in the document.
\usepackage{amsmath}
\usepackage{mathtools}
\graphicspath{ {Images/} } % - Path to where the images are located on your computer. In this case I have a folder (look to the left) "Images" where the images are gathered.
\usepackage{hyperref} % - Package for including hyperlinks in the document.
\usepackage[backend=bibtex,style=numeric,bibencoding=ascii]{biblatex}
 % - Package for the bibliography ("referenser").
\addbibresource{bibliography.bib} % - From where, i.e. which file, the references are taken. The bibliography file is called name.bib; see left column.
\usepackage{graphicx}
\graphicspath{ {./images/} }
\usepackage{minted}

%%%%%%%%%%%%%%%%%%%%%%%%%%%%%%%%%%%%%%%%%%%%%%%%%%%%%%%%%%%%%%
% -               Title and affiliation                    - %
%%%%%%%%%%%%%%%%%%%%%%%%%%%%%%%%%%%%%%%%%%%%%%%%%%%%%%%%%%%%%%

\title {
	Networking
}

\author {
	Philip Wenkel \\
	{\tt phiwen-7@student.ltu.se}
}


\date{\today}

\begin{document}

\maketitle

\newpage
\tableofcontents

\newpage

\section{Clearing a router}

\begin{minted}{bash}
	router# erase startup-config
	router# erase running-config
	router# write erase
	router# reload
\end{minted}

\section{Preparing a standalone switch}
The first commands is to clear and setup the switches in the topology.


\subsection{Delete the VLAN database file}
\begin{minted}{bash}
	switch# delete flash:vlan.dat
\end{minted}

\subsection{Erase the startup config from NVRAM}
\begin{minted}{bash}
	switch# erase startup-config
\end{minted}

\subsection{Reload the device}
\begin{minted}{bash}
	switch# reload
\end{minted}

\section{Clearing a connected switch}
\subsection{Delete the VLAN database file}
\begin{minted}{bash}
	switch# delete vlan.dat
\end{minted}

\subsection{Erase the startup config from NVRAM}
\begin{minted}{bash}
	switch# erase startup-config
\end{minted}

\subsection{Shut down interfaces}
\begin{minted}{bash}
	switch# interface range FastEthernet 0/1 - 24
	switch# shutdown
\end{minted}

\subsection{Remove the VLANs}
\begin{minted}{bash}
	switch# no vlan 2-999
\end{minted}

\subsection{Basic config}
\begin{minted}{bash}
	no ip domain-lookup
	enable secret hej
	banner motd & AUTHORIZED ACCESS ONLY &
	line console 0
	logging synchronous
	exec-timeout 0 0
	password hej2
	login
	exit
	line vty 0 4
	logging synchronous
	exec-timeout 0 0
	password hej3
	login
	exit
\end{minted}

\subsection{Describe an interface}
\begin{minted}{bash}
	int fa0/1
	description R1 LAN
\end{minted}

\subsection{Preparing a layer 3 switch for remote access}

\begin{minted}{bash}
	ip routing
	int vlan 99
	ip address 172.17.99.11 255.255.255.0
	no shutdown
	int fa0/1
	switchport mode access
	switchport access vlan 99
\end{minted}

\subsection{Setting up telnet}
\begin{minted}{bash}
	aaa new-model
	username philip password 0 wenkel
	line vty 0 4
	transport input telnet
\end{minted}

\subsection{Setting up ssh}
\begin{minted}{bash}
	aaa new-model
	username philip password 0 wenkel
	ip domain-name philip.com
	crypto key generate rsa
	ip ssh version 2
	line vty 0 4
	transport input ssh
\end{minted}

In order to connect from my mac, I used the following command 

\begin{minted}{bash}
	ssh -oKexAlgorithms=+diffie-hellman-group1-sha1 -oHostKeyAlgorithms=+ssh-rsa -oCiphers=+aes256-cbc philip@172.17.99.99
\end{minted}

But before this was possible, I had to change the ip address on my network adapter with the following command

\begin{mined}{bash}
	sudo ifconfig en14 inet 172.17.99.101
\end{minted}

\subsection{Configure transparent VTP mode}
\begin{minted}{bash}
	switch# vtp mode transparent
	switch# vtp domain CISCO
\end{minted}

\section{Setting up trunks for DLS}
\begin{minted}{bash}
	switch# int range f0/1-9
	switch# switchport trunk encapsulation dot1q
	switch# switchport mode trunk
\end{minted}

\section{Setting up trunks for ALS}
\begin{minted}{bash}
	switch# int range f0/1-9
	switch# switchport mode trunk
\end{minted}

\section{Change primary and secondary roots}
\begin{minted}{bash}
	switch1# spanning-tree vlan 1 root primary
	switch2# spanning-tree vlan 1 root secondary
\end{minted}

\section{Setting portfast on a switch}
\begin{minted}{bash}
	switch# int fa0/1
	switch# spanning-tree portfast
\end{minted}

\section{Setting port cost}
\begin{minted}{bash}
	switch# int fa0/1
	switch# spanning-tree cost 10
\end{minted}

\section{Assign a root switch for each VLAN}
\begin{minted}{bash}
	switch1# spanning-tree vlan 10 priority 4096
	switch2# spanning-tree vlan 20 priority 4096
\end{minted}

\section{Changing the spanning tree mode to PVRST+}
\begin{minted}{bash}
	switch# spanning-tree mode rapid-pvst
\end{minted}

\section{Changing the spanning tree mode to MST}
\begin{minted}{bash}
	switch# spanning-tree mode mst
\end{minted}

\section{Inter VLAN routing HSRP}

\end{document}

% Kan med fördel indelas i underrubriker; Bakgrund, Problemformulering, Litteraturstudie,
% Syfte och mål, Avgränsningar.
%
% Lite skriv-vett
%
% 1. Använd aktiv form (vattnet strömmade genom röret). Det gör rapporten mer livlig.
% 2. Använd dåtid för observationer mm. Exempelvis ”ökat tryck gav större flöde”.
% 3. Använd nutid för generaliseringar och allmänt giltiga påståenden. Exempelvis
% ”I de flesta fall tillhör problemen kategorin olösbara problem”.
% 4. Undvik strunt, pompösa meningar och alla överdrifter. Uttryck som ``utmärkt
% överenstämmelse'' eller ``fantastisk mätnoggrannhet'' får inte förekomma.
% 5. Samtliga tidigare arbeten som åberopas i rapporten skall refereras.
% 6. Skriv inte rapporten som en berättelse om vad ni gjort.
% 7. Berätta inte om de idéer som inte gav något.
% 8. Var mycket försiktig med negativa kommentarer om det egna arbetet.

%%%%%%%%%%%%%%%%%%%%%%%%%%%%%%%%%%%%%%%%%%%%%%%%%%%%%%%%%%%%%%
% -                       Theory                           - %
%%%%%%%%%%%%%%%%%%%%%%%%%%%%%%%%%%%%%%%%%%%%%%%%%%%%%%%%%%%%%%
% \section{Teori}
% Det fysiska lagret syftar till att sköta kommunikationen över det fysiska mediumet.
% Det kan vara exempelvis över wifi eller via en nätverskabel.
% Data länk lagret sköter kommunikationen via en länk, dvs varje två enheter.
% Nätverkslagret sker mellan två enheter, över en eller flera länkar. I detta 
% lager handlar det om att skicka data till rätt host.
% Transportlagret tar hand om datan mellan processorer av två hostar.
% Sessionslagret har hand om en session mellan hostar. Där handlar det 
% om att hålla sessionen uppdaterad.
% Presentationslagret tar hand om kryptering och komprission av data.
% Det sista lagret, applikationslagret, har hand om datan som själva applikationen skall tolka.

% Beskriv teori, antaganden och annat som ligger till grund för den metod och det arbetssätt som valts. Teorin ska belysa/diskutera/sammanfatta och koppla till problemställningen. Samtliga ekvationer, figurer och tabeller ska numreras i löpande ordning. Figurer och tabeller ska ha en kortfattad text som klart och tydligt anger vad de visar. Alla figurer och tabeller ska hänvisas till i löpande text.
% Exempelvis beskrivs en potensfunktion som
% \begin{equation} \label{epotens}
%     y=Cx^k
% \end{equation}
% där exponenten $k$ bestäms genom att logaritmera ekvation (\ref{epotens})
% \begin{equation}
%     \ln y = \ln C + k \ln x \:\:\:\: \Rightarrow \:\:\:\: z = m + kw
% \end{equation}
% $z = \ln y$ plottas mot $w = \ln x$ och lutningen på grafen ger exponenten $k$ ifall mätdata kan beskrivas som en potensfunktion.

% Med figurer avses bilder, diagram, grafer mm. Det ska inte finnas ytterligare rubrik
% i figuren än den som står i figurtexten. Om ej egentillverkad skall källa anges samt
% tillstånd av ägare erhållas.
%
% Variabler skrivs kursivt både i ekvationerna och i brödtexten. Icke-variabler skrivs på vanligt sätt.
% Ekvationer betraktas som en del i texten. Ekvationsnummer ska skrivas längst ut till höger.

% \subsection{Litteraturstudie}

% Måste inte ha egen rubrik utan kan ingå i teorins löpande text.
% I fall teorikapitlet utelämnas ska litteraturstudie ingå i inledningen.

%%%%%%%%%%%%%%%%%%%%%%%%%%%%%%%%%%%%%%%%%%%%%%%%%%%%%%%%%%%%%%
% -                       Method                           - %
%%%%%%%%%%%%%%%%%%%%%%%%%%%%%%%%%%%%%%%%%%%%%%%%%%%%%%%%%%%%%%
% \section{Metod}

% För att initiera TLS behöver slutenheterna komma överens om olika säkerhetsparametrar.
% Därför börjar klienten skicka ett första meddelande som innehåller dessa.
% Men dessa parametrar måste servern och klienten komma överens om. Det kan
% bland annat bero på att en av dem inte stödjer en viss krypteringsalgoritm.
% När samtliga parametrar har bestämts, kan allt därefter vara krypterat med 
% dessa. 


% Kan i vissa fall delas upp i metodbeskrivning, experimentell uppställning och arbetsgång. Att redogöra för sin metod är viktigt bland annat för att förklara varför den valda metoden ger ett tillförlitligt resultat. Alla antaganden och förenklingar måste anges och motiveras. Definiera matematiska modeller så att andra ingenjörer och forskare kan förstå vad du gjort.
% Exempelvis utnyttjades Microsoft Excel 2013 för att analysera mätresultaten och plotta mätdata.

% Här beskrivs metoden, ofta är det lämpligt att dela upp texten i ett antal underrubriker.
% Använd alltid högst tre rubrik-nivåer.

% \subsection{Experimentell uppställning}

% Alla eventuella försöksuppställningar beskrivs på ett sådant sätt att andra kan upprepa samma försök och verifiera dina resultat. Utnyttja figurer som förenklar din beskrivning.

%%%%%%%%%%%%%%%%%%%%%%%%%%%%%%%%%%%%%%%%%%%%%%%%%%%%%%%%%%%%%%
% -                      Results                           - %
%%%%%%%%%%%%%%%%%%%%%%%%%%%%%%%%%%%%%%%%%%%%%%%%%%%%%%%%%%%%%%
% \section{Resultat}

% All kryptering som sker över nätet skall vara rubust och upgraderingsbar. 
% TLS uppfyller detta kriterie.

% Innehåll: resultat och analys.
% I vissa fall kan man ha ”Resultat och diskussion” som kapitel.

% Detta är förmodligen den största delen av rapporten. Här redovisas resultaten rakt på sak på ett objektivt/neutralt sätt. Ofta är det lämpligt att dela upp texten i ett antal underrubriker. Materialet måste presenteras i logisk ordning, vilket inte behöver vara den ordning i vilken försöket/arbetet har utförts.

% Läsaren skall kunna läsa rapporten utan att behöva bläddra fram och tillbaka. Det ska vara tydligt vad som är data respektive analys av data.
% Visas resultat i tabell- eller figurform så måste kortfattat beskrivas vad man ser i figurerna/tabellerna. De placeras i närheten (efter) där de först refererades.

% Som exempel visas fyra mätningar där variabel, 1, varierades. Resultat visas i tabell \ref{tvariabel123} nedan.

% Det skall alltid finnas en tabelltext som förklarar vad som finns i tabellen. Tabellnummer och text ska stå ovanför tabellen.

% \begin{table}[ht]
% \centering
%     \begin{tabular}{c | c | c}
%         \hline
%         variabel 1 (s) & variabel 2 (m) & variabel 3 (J) \\
%         \hline
%         0,351 &	0,693 &	117 \\
%         0,457 &	1,42 &	170 \\
%         0,873 &	2,54 &	300 \\
%         1,10 &	3,21 &	390 \\
%         \hline
%      \end{tabular} 
% \caption{Förklarande text}
% \label{tvariabel123}
% \end{table}

% Med hjälp av mätvärdena i tabell 1 skapas en produktansats av typen potensfunktion
% \begin{equation}
% t = C L^\alpha \theta^\beta m^\gamma g^\delta
% \end{equation}
% där $C, \alpha, ..., \delta$ är konstanter som ska bestämmas experimentellt.
% Avgiven värmemängd från brödrosten (variabel 3) som funktion av tiden variabel 1 visas i figur \ref{fvariabel3vs1}. Den linjära anpassningen i figuren visar att
% \begin{equation}
% {\rm (variabel \, 3)} = 351,8 \cdot {\rm (variabel \, 1)} - 0,3
% \end{equation}
% och tillförd värmeeffekt till brödrosten bestäms då till 351,8 W.

% Diagram ska ha storhet och enhet på axlarna (SI). Är det tex logaritmerade diagram
% ska de ha storhet (men inte enhet) på axlarna. Figurnummer och text ska stå under
% figur och hänga ihop på samma sida.

% \begin{figure}[ht]
% \begin{center}
%   \includegraphics[width=0.6\textwidth]{fig1.png}
%   \caption{Exempelfigur som visar variabel 3 som funktion av variabel 1. \label{fvariabel3vs1}}
% \end{center}
% \end{figure}

% \subsection{Underrubrik vid behov}

% \subsubsection{Fler underrubriker om så behövs}




%%%%%%%%%%%%%%%%%%%%%%%%%%%%%%%%%%%%%%%%%%%%%%%%%%%%%%%%%%%%%%
% -                      Summary                           - %
%%%%%%%%%%%%%%%%%%%%%%%%%%%%%%%%%%%%%%%%%%%%%%%%%%%%%%%%%%%%%%

% \section{Diskussion och slutsatser}

% Kan delas i separata kapitel: ”Diskussion” respektive ”Slutsatser”. Slutsatser skall vara korta och koncisa.
% Ibland är det lämpligast med indelningen ”Diskussion” samt ”Slutsatser och fortsatt arbete”

% Här diskuteras (vad betyder/medför) resultaten utifrån ett vidare perspektiv och ställs i relation exempelvis till tidigare arbeten, referera i sådant fall till dessa. Utgående härifrån dras nödvändiga slutsatser som ska svara på de mål som angivits och vad resultaten har för relevans. Koppla slutsatser till uppställda mål.

% Diskutera felkällor och osäkerheter.
% Det är även lämpligt att i denna del avsluta med förslag och rekommendationer på fortsatta studier och undersökningar i ämnet. Man kan dela upp diskussion, slutsatser och framtida studier i fristående kapitel.




%%%%%%%%%%%%%%%%%%%%%%%%%%%%%%%%%%%%%%%%%%%%%%%%%%%%%%%%%%%%%%
% -                    Bibliography                        - %
%%%%%%%%%%%%%%%%%%%%%%%%%%%%%%%%%%%%%%%%%%%%%%%%%%%%%%%%%%%%%%
\printbibliography % - Here we say that the bibliography should be printed. The section title "References" is printed automatically.

% I denna del anges de källor du använt i ditt arbete. Ange bara de viktigaste och alla
% referenser i listan måste vara refererade till i texten. I referenslistan får det inte
% förekomma någon referens som är ``allmänt bra att ha'' utan endast de referenser som
% författaren själv använt. Alla referenser ska refereras till i texten.
% OBS! använd originalreferenser. Undvik referenser till webbsidor eftersom de kan försvinna/ändras.
% Ett av de vanligaste är att man skriver
% författarnamnet och sedan referensens publikationsår - det s.k. Harvard systemet.
% Exempel: ... även funnet av Charpak (1983)
% Är det två eller flera författare brukar man skriva
% Två författare: ... även funnet av Charpak och Öqvist (1983).
% Flera författare: ... även funnet av Charpak et al. (1983).
% ( et al. är latin (et alii) och betyder ``och andra'').
% Ett annat sätt att ange en referens är att numrera referenserna i den ordning de dyker
% upp i texten och sortera referenslistan i nummerordning.
% Exempel: ... som Öqvist [3] har funnit, och referensen Öqvist dyker då upp som nummer tre i referenslistan.
% En guide för referenser finns här: http://libguides.ltu.se/skrivaoreferera

% \end{document}


